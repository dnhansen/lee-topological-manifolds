% Document setup
\documentclass[article, a4paper, 11pt, oneside]{memoir}
\usepackage[utf8]{inputenc}
\usepackage[T1]{fontenc}
\usepackage[UKenglish]{babel}

% Document info
\newcommand\doctitle{Concrete topological spaces}
\newcommand\docauthor{Danny Nygård Hansen}

% Formatting and layout
\usepackage[autostyle]{csquotes}
\renewcommand{\mktextelp}{(\textellipsis\unkern)}
\usepackage[final]{microtype}
\usepackage{xcolor}
\frenchspacing
\usepackage{styles/articlepagestyle}
\usepackage{styles/articlesectionstyle}

% Fonts
\usepackage{amssymb}
\usepackage[largesmallcaps,partialup]{kpfonts}
\DeclareSymbolFontAlphabet{\mathrm}{operators} % https://tex.stackexchange.com/questions/40874/kpfonts-siunitx-and-math-alphabets
\linespread{1.06}
% \let\mathfrak\undefined
% \usepackage{eufrak}
\DeclareMathAlphabet\mathfrak{U}{euf}{m}{n}
\SetMathAlphabet\mathfrak{bold}{U}{euf}{b}{n}
% https://tex.stackexchange.com/questions/13815/kpfonts-with-eufrak
\usepackage{inconsolata}

% Hyperlinks
\usepackage{hyperref}
\definecolor{linkcolor}{HTML}{4f4fa3}
\hypersetup{%
	pdftitle=\doctitle,
	pdfauthor=\docauthor,
	colorlinks,
	linkcolor=linkcolor,
	citecolor=linkcolor,
	urlcolor=linkcolor,
	bookmarksnumbered=true
}

% Equation numbering
\numberwithin{equation}{chapter}

% Footnotes
\footmarkstyle{\textsuperscript{#1}\hspace{0.25em}}

% Mathematics
\usepackage{styles/basicmathcommands}
\usepackage{styles/framedtheorems}
\usepackage{styles/topologycommands}
\usepackage{tikz-cd}
\tikzcdset{arrow style=math font} % https://tex.stackexchange.com/questions/300352/equalities-look-broken-with-tikz-cd-and-math-font
\usetikzlibrary{babel}

% Lists
\usepackage{enumitem}
\setenumerate[0]{label=\normalfont(\alph*)}

% Bibliography
\usepackage[backend=biber, style=authoryear, maxcitenames=2, useprefix]{biblatex}
\addbibresource{references.bib}

% Title
\title{\doctitle}
\author{\docauthor}

\begin{document}

\maketitle

\chapter{Introduction}

In this note we elaborate on some of the examples of topological spaces discussed in John M. Lee's \emph{Introduction to Topological Manifolds}, henceforth denoted [TM]. References are placed in brackets at the end of each text unit. All references are to [TM].

We follow the notation used in [TM], except that we write $X \homeo Y $ to denote that topological spaces $X$ and $Y$ are homeomorphic. Furthermore, norms are denoted $\norm{\,\cdot\,}$.

Furthermore, we make frequent use of a few central results without explicit reference: First of all the closed map lemma [TM, Lemma~4.50], which says that continuous maps from compact spaces into Hausdorff spaces are closed. Secondly the uniqueness of quotient spaces [TM, Theorem~3.75], which says that if two quotient maps make the same identifications, then the associated quotient spaces are homeomorphic.


\chapter{Basic spaces}

\begin{example}[$\ball^n \homeo \reals^n$]
    The unit ball $\ball^n$ is homemorphic to $\reals^n$ through the map $F \colon \ball^n \to \reals^n$ given by
    %
    \begin{equation*}
        F(x) = \frac{x}{1 - \norm{x}}.
    \end{equation*}
    %
    Direct computation shows that $F$ has inverse $G \colon \reals^n \to \ball^n$ given by
    %
    \begin{equation*}
        G(y) = \frac{y}{1 + \norm{y}}.
    \end{equation*}
    %
    [2.25]
\end{example}


\begin{example}
    The $2$-sphere $\sphere^2$ is homemorphic to the cubic surface $C \subseteq \reals^3$ with side length $2$ centered at the origin. The map $\phi \colon C \to \sphere^2$ that normalises a vector in $C$ is a homeomorphism: it and its inverse are easy to write down explicitly. [2.26]
\end{example}


\begin{example}[Graphs of continuous functions]
    Let $U \subseteq \reals^n$ be open, and let $f \colon U \to \reals^k$ be continuous. Then $U$ is homeomorphic to the graph $\graph{f}$ of $f$ through the homeomorphism $\Phi_f \colon U \to \reals^{n+k}$ given by $\Phi_f(x) = (x, f(x))$. Thus $\graph{f}$ is a manifold. [3.20]
\end{example}


\begin{example}[Spheres]
    The sphere $\sphere^n$ is a manifold. We produce two sets of charts:

    For $i = 1, \ldots, n+1$, let
    %
    \begin{equation*}
        U_i^\pm = \set{x \in \reals^{n+1}}{{\pm x_i} > 0}.
    \end{equation*}
    %
    These sets cover $\sphere^n$. On $U_i^\pm$ we can solve the equation $\norm{x} = 1$ and find that $x_i$ is a continuous function of the other coordinates. Thus $\sphere^n \intersect U_i^\pm$ is the graph of a continuous function.

    Stereographic projection: Let $N = (0, \ldots, 0, 1)$ be the \textquote{north pole} in $\sphere^n$, and define the stereographic projection $\sigma \colon \sphere^n \setminus \{N\} \to \reals^n$ as the map that sends $x \in \sphere^n \setminus \{N\}$ to a point $u \in \reals^n$ such that $U = (u,0)$ is the intersection of the line through $N$ and $x$ with the subspace where $x_{n+1} = 0$. Note that $u = \sigma(x)$ can be written $U - N = \lambda (x - N)$ for some $\lambda \in \reals$. Solving this for $\lambda$ gives an explicit form for $\sigma$, whose inverse can also be found explicitly. In particular, this provides a Euclidean neighbourhood of every point of $\sphere^n$ except for $N$. [3.21]
\end{example}


\chapter{Quotient spaces}

\begin{example}[$\sphere^1$ as a quotient space]
    Let $I = [0,1]$, and define an equivalence relation $\sim$ on $I$ by identifying $0$ and $1$. By the closed map lemma, the map $\omega \colon I \to \sphere^1$ given by $\omega(s) = \e^{2\pi \iu s}$ is a quotient map. Since $\omega$ makes the same identifications as $\sim$, $I/{\sim}$ is homeomorphic to $\sphere^1$. [3.47, 3.76, 4.51]
\end{example}


\begin{example}[Spheres as quotients]
    Define a map $q \colon \reals^{n+1} \setminus \{0\} \to \sphere^n$ by $q(x) = x/\norm{x}$. This is continuous and surjective, and its fibres are open rays in $\reals^{n+1} \setminus \{0\}$. Hence the saturated sets are unions of open rays, and $q$ sends these to open subsets of of $\sphere^n$, so $q$ is a quotient map. Thus $\sphere^n$ is obtained from $\reals^{n+1} \setminus \{0\}$ by collapsing open rays to a point. [3.64]
\end{example}


\begin{example}[Tori]
    Define an equivalence relation on the square $I \prod I$ by letting $(x,0) \sim (x,1)$ and $(0,y) \sim (1,y)$ for all $x,y \in I$. We claim that this quotient space is homeomorphic to the torus $\torus^2$. Define a map $q \colon I \prod I \to \torus^2$ by $q(u,v) = (\e^{2\pi\iu u}, \e^{2\pi\iu v})$. This is a quotient map by the closed map lemma, and since it makes the same identifications as the equivalence relation $\sim$, the quotient $(I \prod I)/{\sim}$ is homeomorphic to $\torus^2$.
    
    Now consider the doughnut surface $D \subseteq \reals^3$ obtained by revolving the circle $(x-2)^2 + z^2 = 1$ around the $z$-axis. It is characterised by the equation $(r-2)^2 + z^2 = 1$, where $r = \sqrt{x^2 + y^2}$. Thus there is an angle $\phi$ such that $z = \sin\phi$ and $r-2 = \cos\phi$. It follows that
    %
    \begin{equation*}
        x = r\cos\theta = (2 + \cos\phi) \cos\theta
        \quad \text{and} \quad
        y = r\sin\theta = (2 + \cos\phi) \sin\theta,
    \end{equation*}
    %
    for some angle $\theta$. Making the substitutions $\phi = 2\pi u$ and $\theta = 2\pi v$ yields a surjective map $F \colon \reals^2 \to D$ given by
    %
    \begin{equation*}
        F(u,v) = \bigl(
            (2 + \cos 2\pi u) \cos 2\pi v,
            (2 + \cos 2\pi u) \sin 2\pi v,
            \sin 2\pi u
        \bigr).
    \end{equation*}
    %
    This map is clearly not injective, but restricting it to the square $I \prod I$ yields a surjective map that makes the same identifications as $q$. By the closed map lemma it is a quotient map, so $D$ and $\torus^2$ are homeomorphic. [3.22, 3.49, 4.52, 4.53]
\end{example}


\begin{example}[Real projective space]
    Let $\projspace^n$ be the set of one-dimensional (linear) subspaces of $\reals^{n+1}$, and let $q \colon \reals^{n+1} \setminus \{0\} \to \projspace^n$ send points to their span. Give $\projspace^n$ the quotient topology with respect to $q$.

    Alternatively, define an equivalence relation on $\reals^{n+1} \setminus \{0\}$ by declaring that $x$ and $y$ be equivalent if $x = \lambda y$ for some $y \in \reals \setminus \{0\}$. Except for the point $0$, the equivalence classes are the fibres of $q$.

    We can also represent $\projspace^n$ as a quotient of $\sphere^n$ by identifying antipodal points. Let $\sim$ denote this equivalence relation, and let $p \colon \sphere^n \to \sphere^n/{\sim}$ be the associated quotient map. Consider the composite map
    %
    \begin{equation*}
        \begin{tikzcd}
            \sphere^n
                \ar[r, hook, "\iota"]
            & \reals^{n+1} \setminus \{0\}
                \ar[r, "q"]
            & \projspace^n,
        \end{tikzcd}
    \end{equation*}
    %
    where $\iota$ is the inclusion. The composition $q \circ \iota$ is a quotient map by the closed map lemma, and it makes the same identifications as $p$, so $\projspace^n$ is homeomorphic to $\sphere^n/{\sim}$. This also shows that $\projspace^n$ is compact. [3.51, 4.54]
\end{example}


\begin{example}[Collapsing $\boundary\clball^n$ to a point]
    Let $\clball^n/\sphere^{n-1}$ be the quotient space obtained by collapsing the boundary of $\clball^n$ to a point. We show that this is homeomorphic to $\sphere^n$. To this end, let $q \colon \clball^n \to \sphere^n$ be the map given by
    %
    \begin{equation*}
        q(x) = \bigl(
            2 \sqrt{1 - \norm{x}^2} x,
            2 \norm{x}^2 - 1
        \bigr).
    \end{equation*}
    %
    Computing the norm of $q(x)$ shows that this is well-defined. The map is surjective and continuous, and a quotient map by the closed map lemma. Notice that it is injective on $\ball^n$ and constant on $\boundary\ball^n$, so it makes the same identifications as the quotient map $\clball^n \to \clball^n/\sphere^{n-1}$. Hence this quotient space is homeomorphic to $\sphere^n$. [3.52, 4.55]
\end{example}


\begin{example}[$\cone\sphere^n \homeo \clball^{n+1}$]
    The map $F \colon \sphere^n \prod I \to \clball^{n+1}$ defined by $F(x,s) = sx$ is continuous and surjective, and by the closed map lemma it is a quotient map. It maps $\sphere^n \prod \{0\}$ to $0 \in \clball^{n+1}$ and is injective elsewhere, so it makes the same identifications as the quotient map $\sphere^n \prod I \to \cone\sphere^n$. Hence $\cone\sphere^n$ is homeomorphic to $\clball^{n+1}$. [3.65, 4.56]
\end{example}


\chapter{Adjunction spaces}

\begin{example}
    Let $(X,x)$ and $(Y,y)$ be pointed topological spaces. Let $A = \{y\}$ and define $f \colon A \to X$ by $f(y) = x$. Then the adjunction space $X \union_f Y$ and the wedge sum $X \wedgesum Y$ are identical. [3.78(a)]
\end{example}

\begin{example}[Gluing two balls together]
    Let $A = \sphere^1 \subseteq \clball^2$, and let $f \colon A \hookrightarrow \clball^2$ be the inclusion map. Then the adjunction space $\clball^2 \union_f \clball^2$ is homeomorphic to $\sphere^2$: Let $q \colon \clball^2 \disunion \clball^2 \to \sphere^2$ be the map that sends the first copy of $\clball^2$ to the upper hemisphere of $\sphere^2$ and the second copy to the lower hemisphere. This makes the same identifications as $f$, so the adjunction space is homeomorphic to $\sphere^2$. [3.78(b)]
\end{example}





\end{document}