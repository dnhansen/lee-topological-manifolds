% Document setup
\documentclass[article, a4paper, 11pt, oneside]{memoir}
\usepackage[utf8]{inputenc}
\usepackage[T1]{fontenc}
\usepackage[UKenglish]{babel}

% Document info
\newcommand\doctitle{Lee: \emph{Introduction to Topological Manifolds}}
\newcommand\docauthor{Danny Nygård Hansen}

% Formatting and layout
\usepackage[autostyle]{csquotes}
\renewcommand{\mktextelp}{(\textellipsis\unkern)}
\usepackage[final]{microtype}
\usepackage{xcolor}
\frenchspacing
\usepackage{styles/articlepagestyle}
\usepackage{styles/articlesectionstyle}

% Fonts
\usepackage{amssymb}
\usepackage[largesmallcaps,partialup]{kpfonts}
\DeclareSymbolFontAlphabet{\mathrm}{operators} % https://tex.stackexchange.com/questions/40874/kpfonts-siunitx-and-math-alphabets
\linespread{1.06}
% \let\mathfrak\undefined
% \usepackage{eufrak}
\DeclareMathAlphabet\mathfrak{U}{euf}{m}{n}
\SetMathAlphabet\mathfrak{bold}{U}{euf}{b}{n}
% https://tex.stackexchange.com/questions/13815/kpfonts-with-eufrak
\usepackage{inconsolata}

% Hyperlinks
\usepackage{hyperref}
\definecolor{linkcolor}{HTML}{4f4fa3}
\hypersetup{%
	pdftitle=\doctitle,
	pdfauthor=\docauthor,
	colorlinks,
	linkcolor=linkcolor,
	citecolor=linkcolor,
	urlcolor=linkcolor,
	bookmarksnumbered=true
}

% Equation numbering
\numberwithin{equation}{chapter}

% Footnotes
\footmarkstyle{\textsuperscript{#1}\hspace{0.25em}}

% Mathematics
\usepackage{styles/basicmathcommands}
\usepackage{styles/framedtheorems}
\usepackage{styles/topologycommands}
\usepackage{tikz-cd}
\tikzcdset{arrow style=math font} % https://tex.stackexchange.com/questions/300352/equalities-look-broken-with-tikz-cd-and-math-font
\usetikzlibrary{babel}

% Lists
\usepackage{enumitem}
\setenumerate[0]{label=\normalfont(\alph*)}

% Bibliography
\usepackage[backend=biber, style=authoryear, maxcitenames=2, useprefix]{biblatex}
\addbibresource{references.bib}

% Title
\title{\doctitle}
\author{\docauthor}


%% Framed exercise environment

\mdfdefinestyle{swannexercise}{%
    skipabove=0.5em plus 0.4em minus 0.2em,
	skipbelow=0.5em plus 0.4em minus 0.2em,
	leftmargin=-5pt,
	rightmargin=-5pt,
	innerleftmargin=5pt,
	innerrightmargin=5pt,
	innertopmargin=5pt,
	innerbottommargin=4pt,
	linewidth=0pt,
	splittopskip=1.2em minus 0.2em,
	splitbottomskip=0.5em plus 0.2em minus 0.1em,
	backgroundcolor=backgroundcolor,
	frametitlebackgroundcolor=titlecolor,
	frametitlefont={\scshape},
    theoremseparator={\space\thechapter},
    theoremspace={.},
	frametitleaboveskip=3pt,
	frametitlebelowskip=2pt
}

\mdtheorem[style=swannexercise]{exerciseframed}{Exercise}

\let\oldexerciseframed\exerciseframed
\renewcommand{\exerciseframed}{%
  \crefalias{theorem}{exerciseframed}%
  \oldexerciseframed}


\mdtheorem[style=swannexercise]{problemframed}{Problem}

\let\oldproblemframed\problemframed
\renewcommand{\problemframed}{%
    \crefalias{theorem}{problemframed}%
    \oldproblemframed}

\makeatother



\theoremstyle{nonumberplain}
\theoremsymbol{\ensuremath{\square}}
\newtheorem{solution}{Solution}

\let\oldsolution\solution
\renewcommand{\solution}{%
  \crefalias{theorem}{solution}%
  \oldsolution}

\newcommand{\solutionlabelfont}[1]{{\normalfont\color{linkcolor}#1}}
\newlist{solutionsec}{enumerate}{1}
\setlist[solutionsec]{leftmargin=0pt, parsep=0pt, listparindent=\parindent, font=\solutionlabelfont, label=(\alph*), labelsep=0pt, labelwidth=20pt, itemindent=20pt, align=left, itemsep=10pt}

\newenvironment{displaytheorem}{%
	\begin{displayquote}\itshape%
}{%
	\end{displayquote}%
}


\newcommand{\calB}{\mathcal{B}}
\newcommand{\calV}{\mathcal{V}}
\newcommand{\calU}{\mathcal{U}}
\newcommand{\calF}{\mathcal{F}}
\newcommand{\calE}{\mathcal{E}}
\newcommand{\calT}{\mathcal{T}}
\newcommand{\bbF}{\mathbb{F}}

\begin{document}

\maketitle

\addtocounter{chapter}{1}
\chapter{Topological Spaces}

\begin{exerciseframed*}[32]
    \begin{enumerate}
        \item Every homeomorphism is a local homeomorphism.
        \item Every local homeomorphism is continuous and open.
        \item Every bijective local homeomorphism is a homeomorphism.
    \end{enumerate}
\end{exerciseframed*}

\begin{solution}
\begin{solutionsec}
    \item This is obvious since the domain is an open neighbourhood of each point, and this is mapped onto the codomain which is also open.

    \item Continuity follows from Exercise~5.3 and Problem~5.6. To prove openness, let $f \colon X \to Y$ be a local homeomorphism and $V \subseteq X$ an open set. Let $\calU$ be a cover of $V$ of open sets in accordance with the definition of local homeomorphisms. Then $V \intersect U$ is open in $U$ for all $U \in \calU$, and since $f|_U$ is a homeomorphism $f(V \intersect U)$ is also open in $f(U)$, hence in $Y$. Furthermore,
    %
    \begin{equation*}
        f(V)
            = \bigunion_{U \in \calU} f(V \intersect U),
    \end{equation*}
    %
    so $f(V)$ is a union of open sets in $Y$.

    \item This is obvious from (b).
\end{solutionsec}
\end{solution}


\section*{Problems}

\begin{problemframed*}[6]
    Suppose $X$ and $Y$ are topological spaces, and $f \colon X \to Y$ is any map.
    %
    \begin{enumerate}
        \item $f$ is continuous if and only if $f(\closure{A}) \subseteq \closure{f(A)}$ for all $A \subseteq X$.
        \item $f$ is closed if and only if $f(\closure{A}) \supseteq \closure{f(A)}$ for all $A \subseteq X$.
    \end{enumerate}
\end{problemframed*}

\begin{solution}
\begin{solutionsec}
    \item First assume that $f$ is continuous, and let $A \subseteq X$. Then $A \subseteq f\preim(\closure{f(A)})$, which implies that $\closure{A} \subseteq f\preim(\closure{f(A)})$. This is equivalent to the property above.

    Conversely, let $F \subseteq Y$ be closed. Then
    %
    \begin{equation*}
        f \bigl( \closure{f\preim(F)} \bigr)
            \subseteq \overline{f(f\preim(F))}
            \subseteq \closure{F}
            = F,
    \end{equation*}
    %
    showing that $f$ is continuous.

    \item The \enquote{if} implication is obvious, so we prove the converse. Let $F \subseteq X$ be closed. Then
    %
    \begin{equation*}
        f(F)
            = f(\closure{F})
            \supseteq \closure{f(F)},
    \end{equation*}
    %
    which shows that $f(F) = \closure{f(F)}$, so $f(F)$ is closed.
\end{solutionsec}
\end{solution}


\begin{problemframed*}[21]
    Show that every locally Euclidean space is first countable.
\end{problemframed*}

\begin{solution}
    Let $M$ be locally Euclidean of dimension $n$. If $p \in M$, then $p$ lies in the domain of a chart $(U,\phi)$. For simplicity we may let $U$ be homeomorphic to $\ball^n$. For $r \in \rationals \intersect (0,1]$ let $B_r$ be the preimage under $\phi$ of the ball $B_r(0) \subseteq \reals^n$. We claim that the $B_r$ constitute a neighbourhood basis at $p$.

    Let $V$ be an open neighbourhood of $p$. By intersecting it with $U$ we may assume that $V \subseteq U$. Then $\phi(V)$ is open in $\ball^n$ and hence contains $B_r(0)$ for some $r \in \rationals \intersect (0,1]$. But then $V$ contains $B_r$.
\end{solution}


\begin{problemframed*}[23]
    Show that every manifold has a basis of coordinate balls.
\end{problemframed*}
%
[TODO: Is being locally Euclidean not enough?]

\begin{solution}
    Let $M$ be an $n$-manifold, and let $U \subseteq M$ be open. If $p \in U$, then $p$ lies in the domain of a chart. By intersecting $U$ with this chart domain we may assume that $U$ is itself the domain of a chart with coordinate map $\phi$. Since $\phi(U) \subseteq \reals^n$ is open it contains an open ball $B$ with $p \in B$. Notice that $\phi$ restricts to a homeomorphism $\phi\preim(B) \to B$, so this preimage is a coordinate ball lying in $U$ and containing $p$.
\end{solution}


\begin{problemframed*}[25]
    If $M$ is an $n$-dimensional manifold with boundary, then $\interiorop M$ is an open subset of $M$, which is itself an $n$-dimensional manifold without boundary.
\end{problemframed*}

\begin{solution}
    It suffices to show that $\interiorop M$ is open in $M$. To this end, let $p \in \interiorop M$, and let $(U,\phi)$ be an interior chart with $p \in U$. If $q \in U$, then $q$ also lies in the domain of an interior chart, namely $(U,\phi)$. Thus $q \in \interiorop M$, so $U \subseteq \interiorop M$.
\end{solution}


\chapter{New Spaces from Old}

\section*{General notes}

\begin{remark}
    We rephrase Proposition~3.56 and its proof. First note that Problem~2.19 (which is used in its proof) implies that a locally Euclidean Lindelöf space is second countable. Next let $X$ be a Lindelöf space (so for instance a second countable space), $Y$ a locally Euclidean space, and let $f \colon X \to Y$ be a continuous surjection. Then $Y$ is also Lindelöf, hence second countable.

    In particular, the hypothesis that $f$ be a quotient map is superfluous.
\end{remark}


\begin{remark}
    We comment on the proof of Proposition~3.67, which shows that a continuous open/closed surjection is a quotient map. Let $q \colon X \to Y$ be such a map, and let $U \subseteq Y$. It suffices to show that if $q\preim(U)$ is open/closed, then so is $U$. But since $q$ is open/closed, so is $q(q\preim(U)) = U$, where the equality follows since $q$ is surjective.

    In other words, there is no need to appeal to Proposition~3.60.
\end{remark}


\section*{Exercises}

\begin{exerciseframed*}[32]
    Let $\{ X_\alpha \}_{\alpha \in A}$ be a collection of topological spaces
    %
    \begin{enumerate}
        \item For any $\beta \in A$ and any $x_\alpha \in X_\alpha$ for $\alpha \neq \beta$, define a map $f \colon X_\beta \to \bigprod_{\alpha \in A} X_\alpha$ by $f(x) = (x_\alpha)_{\alpha \in A}$, where $x_\beta = x$. Then $f$ is a topological embedding.

        \item Each canonical projection $\pi_\beta \colon \bigprod_{\alpha \in A} X_\alpha \to X_\beta$ is an open map.
    \end{enumerate}
\end{exerciseframed*}

\begin{solution}
\begin{solutionsec}
    \item It is easy to see by looking at a basis for the product space that $f$ is continuous. By Problem~3.13 it suffices to show that $f$ has a continuous left inverse. But the projection $\pi_\beta$ is a left inverse for $f$, and this is continuous.

    \item It suffices to show that $\pi_\beta(U)$ is open for all elements $U$ in a basis for the product topology. Hence assume that $U$ is on the form $\bigprod_{\alpha \in A} U_\alpha$, where $U_\alpha$ is open in $X_\alpha$, and $U_\alpha = X_\alpha$ except for finitely many $\alpha$. But then $\pi_\beta(U) = U_\beta$, which is open.
\end{solutionsec}
\end{solution}


\begin{exerciseframed*}[55]
    Show that every wedge sum of Hausdorff spaces is Hausdorff.
\end{exerciseframed*}

\begin{solution}
    Let $\{X_\alpha\}_{\alpha \in A}$ be a collection of Hausdorff spaces, denote the base point of $X_\alpha$ by $p_\alpha$, and let $x$ and $y$ be distinct points in their wedge sum.

    First assume that neither point is the base point, and that $x,y \in X_\alpha$ for some $\alpha \in A$. Picking disjoint open neighbourhoods $U,V \subseteq X_\alpha$ of $x$ and $y$ yields disjoint open neighbourhoods $U \setminus \{p_\alpha\}$ and $V \setminus \{p_\alpha\}$ of $x$ and $y$ in $X_\alpha$. These are clearly also disjoint and open in $\bigvee_{\alpha \in A} X_\alpha$.

    Next assume that neither point is the base point, and that $x \in X_\alpha$ and $y \in X_\beta$ for some $\alpha \neq \beta$. In this case the sets $X_\alpha \setminus \{p_\alpha\}$ and $X_\beta \setminus \{p_\beta\}$ work.

    Finally assume that $x$ is the base point and that $y \in X_\beta$. Let $x \in U$ and $y \in V$ in $X_\beta$. The set
    %
    \begin{equation*}
        U \vee \bigjoin_{\alpha \neq \beta} X_\alpha
    \end{equation*}
    %
    is an open neighbourhood of $x$ disjoint from $V$.
\end{solution}

\begin{exerciseframed*}[59]
    Let $f \colon X \to Y$ be any map. For a subset $A \subseteq X$, show that the following are equivalent:
    %
    \begin{enumerate}
        \item $A$ is saturated.
        \item $A = f\preim(f(A))$.
        \item $A$ is a union of fibres.
        \item If $x \in A$, then $f(x) = f(x')$ implies that $x' \in A$, for all $x' \in X$.
    \end{enumerate}
\end{exerciseframed*}

\begin{solution}
\begin{proofsec}
    \item[(a) $\Leftrightarrow$ (b)]
    First assume that $A$ is saturated, so that $A = f\preim(B)$ for some $B \subseteq Y$. Then $f(A) \subseteq B$, so $f\preim(f(A)) \subseteq f\preim(B) = A$, and the opposite inclusion always holds. The opposite implication is obvious.

    \item[(a) $\Leftrightarrow$ (c)]
    Simply notice that
    %
    \begin{equation*}
        f\preim(B)
            = f\preim \biggl( \bigunion_{y \in B} \{y\} \biggr)
            = \bigunion_{y \in B} f\preim(y)
    \end{equation*}
    %
    for any $B \subseteq Y$, so $A$ is on the form $f\preim(B)$ if and only if it is a union of fibres.

    \item[(c) $\Leftrightarrow$ (d)]
    First assume that $A$ is a union if fibres, and let $x \in A$. If $f(x) = f(x')$ for some $x' \in X$, then $x$ and $x'$ lie in the same fibre, and this fibre is either contained entirely in $A$ or is disjoint from $A$. Hence $x' \in A$.
    
    Conversely, if $x \in A$ then $f\preim(x) \subseteq A$, since $f(x) = f(x')$ for all $x'$ in this preimage.
\end{proofsec}
\end{solution}


\begin{exerciseframed*}[61]
    A continuous surjective map $q \colon X \to Y$ is a quotient map if and only if it takes saturated open subsets to open subsets, or saturated closed subsets to closed subsets.
\end{exerciseframed*}

\begin{solution}
    First assume that $q$ is a quotient map, and let $U \subseteq X$ be a saturated open subset. Then since $U = q\preim(q(U))$, the set $U$ is open if and only if $q(U)$ is open.

    Conversely, assume that $q$ takes saturated open subsets to open subsets, and let $V \subseteq Y$. We need to show that $V$ is open if and only if $q\preim(V)$ is open. The \enquote{only if} part follows since $q$ is continuous. For the \enquote{if} part, assume that $q\preim(V)$ is open and notice that it is saturated. But then $q(q\preim(V)) = V$ is also open, where the equality follows since $q$ is surjective.

    The case where $q$ takes saturated closed subsets to closed subsets follows similarly (replace \enquote{open} with \enquote{closed} throughout).
\end{solution}


\begin{exerciseframed*}[62]
    Properties of quotient maps.
    %
    \begin{enumerate}
        \item Any composition of quotient maps is a quotient map.
        \item An injective quotient map is a homeomorphism.
        \item If $q \colon X \to Y$ is a quotient map, a subset $F \subseteq Y$ is closed if and only if $q\preim(F)$ is closed in $X$.
        \item If $q \colon X \to Y$ is a quotient map and $U \subseteq X$ is a saturated open or closed subset, then the restriction $q|_U \colon U \to q(U)$ is a quotient map.
        \item If $\{q_\alpha \colon X_\alpha \to Y_\alpha\}_{\alpha \in A}$ is an indexed family of quotient maps, then the map $q \colon \coprod_{\alpha \in A} X_\alpha \to \coprod_{\alpha \in A} Y_\alpha$ whose restriction to each $X_\alpha$ is equal to $q_\alpha$ is a quotient map.
    \end{enumerate}
\end{exerciseframed*}

\begin{solution}
\begin{solutionsec}
    \item This follows from the fact that final topologies compose.

    \item Let $q \colon X \to Y$ be an injective quotient map. Since $q$ is already continuous and surjective, it suffices to show that it is open. For any subset $U \subseteq X$ we have $U = q\preim(q(U))$ since $q$ is injective, so $U$ is open if and only if $q(U)$ is. Hence $q$ is open.
    
    \item This follows easily from the definition of the quotient topology by taking complements.

    \item Assume that $U$ is open (the case where $U$ is closed is similar). By Proposition~3.60 (or Exercise~3.61) it suffices to show that $q|_U$ takes saturated open subsets of $U$ to open subsets, so let $V \subseteq U$ be open. Then $V$ is also open in $X$, and $q(V)$ is open in $Y$. But then $q(V)$ is also open in $q(U)$ as desired.

    \item This also follows from the fact that final topologies compose. To be explicit, consider for each $\beta \in A$ the diagram
    %
    \begin{equation*}
        \begin{tikzcd}[column sep=small]
            X_\beta
                \ar[rr, "q_\beta"]
                \ar[dr, "\iota_{X_\beta}", swap]
            && Y_\beta
                \ar[dr, "\iota_{Y_\beta}"] \\
            & \coprod\limits_{\alpha \in A} X_\alpha
                \ar[rr, "q", swap]
            && \coprod\limits_{\alpha \in A} Y_\alpha
        \end{tikzcd}
    \end{equation*}
    %
    Each $Y_\beta$ has the final topology coinduced by $q_\beta$, and $\coprod_{\alpha \in A} Y_\alpha$ has the final topology coinduced by the maps $\iota_{Y_\beta}$. Since final topologies compose, this also has the final topology induced by the maps $\iota_{Y_\beta} \circ q_\beta$. But since the above diagram commutes, this is the same as the final topology coinduced by the maps $q \circ \iota_{X_\beta}$. But since $\coprod_{\alpha \in A} X_\alpha$ itself carries the final topology coinduced by the $\iota_{X_\beta}$, the topology on $\coprod_{\alpha \in A} Y_\alpha$ is the same as the final topology coinduced by the map $q$.
\end{solutionsec}
\end{solution}


\section*{Problems}

\begin{problemframed*}[1]
    Suppose $M$ is an $n$-dimensional manifold with boundary. Show that $\boundary M$ is an $(n-1)$-manifold (without boundary) when endowed with the subspace topology. (We may assume invariance of the boundary.)
\end{problemframed*}

\newcommand{\halfspace}{\mathbb{H}}

\begin{solution}
    Let $p \in \boundary M$. Then $p$ lies in the domain of a boundary chart $(U,\phi)$, and $\phi(p) \in \boundary \halfspace^n$. We claim that $\phi$ maps $U \intersect \boundary M$ into $\boundary \halfspace^n$. If $q \in U \intersect \boundary M$, then $q$ does not lie in $\interiorop M$ (since we are assuming invariance of the boundary). If we had $\phi(q) \in \interiorop \halfspace^n$, then we could restrict the domain of $U$ to a smaller open set $U'$ such that we still had $q \in U'$, and such that $\phi(U')$ was an open subset of $\interiorop\halfspace^n$. (For instance, we could simply remove $\boundary\halfspace^n$ from the codomain and restrict the domain accordingly.) But this is impossible since $q$ is not an interior point, so we must have $\phi(q) \in \boundary\halfspace^n$.

    Hence $\phi$ restricted to $U \intersect \boundary M$ is a chart from $\boundary M$ to $\boundary\halfspace^n \cong \reals^{n-1}$.
\end{solution}


\begin{problemframed*}[13]
    Suppose $X$ and $Y$ are topological spaces and $f \colon X \to Y$ is a continuous map. Prove the following:
    %
    \begin{enumerate}
        \item If $f$ admits a continuous left inverse, it is a topological embedding.
        \item If $f$ admits a continuous right inverse, it is a quotient map.
        \item TODO
    \end{enumerate}
\end{problemframed*}

\begin{solution}
\begin{solutionsec}
    \item Let $g \colon Y \to X$ be a continuous left inverse of $f$, and define $\tilde{f} \colon X \to f(X)$ by $\tilde{f}(x) = f(x)$. Then $g|_{f(X)}$ is continuous and a left inverse of $\tilde{f}$. But since $\tilde{f}$ is bijective it has a unique (two-sided) inverse, namely $g|_{f(X)}$, and since this is also continuous $\tilde{f}$ is a homeomorphism.

    \item Let $g \colon Y \to X$ be a continuous right inverse of $f$, and let $U \subseteq Y$ be such that $f\preim(U)$ is open. Then
    %
    \begin{equation*}
        g\preim(f\preim(U))
            = (f \circ g)\preim(U)
            = U
    \end{equation*}
    %
    is also open, so $f$ is a quotient map.

    \item
\end{solutionsec}
\end{solution}

\begin{remark}
    I would have liked for split mono- and epimorphisms to be open and closed, but I'm not sure this is the case. Instead, this problem and Proposition~3.69 seem to provide independent criteria under which a continuous map $f$ is an embedding or a quotient map: In either case $f$ must be injective/surjective. But then we further assume either that $f$ splits, or that $f$ is open/closed. In other words, the hypothesis that $f$ splits can be replaced with the weaker hypothesis that $f$ is injective/surjective, but then we must further assume that $f$ is open or closed.
\end{remark}


\addtocounter{chapter}{1}
\chapter{Cell Complexes}

\section*{General notes}

\begin{remarkbreak}[Coherent topologies]
    \label{rem:coherent-topologies}
    Let $(X,\calT)$ be a topological space, and let $\calB$ be a collection of subspaces of $X$ whose union is $X$. We claim that $\calT$ is coherent with $\calB$ if and only if $\calT$ is the final topology coinduced by the inclusion maps $\iota_B \colon B \to X$ for all $B \in \calB$. For this topology has the property that a set $U \subseteq X$ is open if and only if $\iota_B\preim(U) = U \intersect B$ is open in $B$ for all $B \in \calB$.

    This is precisely Problem~5.5.
\end{remarkbreak}


\begin{remark}
    We comment on Proposition~5.4, the claim that Hausdorff spaces equipped with a locally finite cell depomposition is a CW complex.

    Let $\calE$ be the cell decomposition in question, and let $\closure{\calE} = \set{\closure{e}}{e \in \calE}$. Then $\closure{\calE}$ is a locally finite closed cover of $X$, so Problem~5.6(b) shows that the topology of $X$ is coherent with $\closure{\calE}$.
\end{remark}


\section*{Exercises}

\begin{exerciseframed*}[3]
    Suppose $X$ is a topological space whose topology is coherent with a family $\calB$ of subspaces.
    %
    \begin{enumerate}
        \item If $Y$ is another topological space, then a map $f \colon X \to Y$ is continuous if and only if $f|_B$ is continuous for every $B \in \calB$.
        \item The map $q \colon \coprod_{B\in\calB} B \to X$ induced by the inclusion of each set $B \hookrightarrow X$ is a quotient map.
    \end{enumerate}
\end{exerciseframed*}

\begin{solution}
\begin{solutionsec}
    \item Let $V \subseteq Y$. Then $f\preim(V)$ is open if and only if $(f|_B)\preim(V) = f\preim(V) \intersect B$ is open in $B$ for all $B \in \calB$. But this precisely expresses that each $f|_B$ is continuous, so the claim follows.

    Alternatively, since $f|_B = \iota_B \circ f$, this precisely expresses the universal property of the final topology induced by the inclusion maps $\iota_B$, so this follows from \cref{rem:coherent-topologies}.

    \item Notice that $q\preim(U) = U \intersect B$ for all $U \subseteq X$. Since the topology on $X$ is coherent with $\calB$, the set $U$ is open if and only if $U \intersect B$ is open for all $B \in \calB$. But this precisely expresses that $q$ is a quotient map.

    Alternatively, this follows since $X$ has the final topology induced by the inclusion maps, but the disjoint union $\coprod_{B\in\calB} B$ also has a final topology, and final topologies compose.
\end{solutionsec}
\end{solution}


\section*{Problems}

\begin{problemframed*}[5]
    Suppose $X$ is a topological space and $\{X_\alpha\}$ is a family of subspaces whose union in $X$. Show that the topology of $X$ is coherent with the subspaces $\{X_\alpha\}$ if and only if it is the finest topology on $X$ for which all of the inclusion maps $i_\alpha \colon X_\alpha \hookrightarrow X$ are continuous.
\end{problemframed*}

\begin{solution}
    This follows immediately from the fact that $i_\alpha\preim(U) = U \intersect X_\alpha$ for all $U \subseteq X$.
\end{solution}


\begin{problemframed*}[6]
    Suppose $X$ is a topological space. Show that the topology of $X$ is coherent with each of the following collections of subspaces of $X$:
    %
    \begin{enumerate}
        \item Any open over of $X$.
        \item Any locally finite closed cover of $X$.
    \end{enumerate}
\end{problemframed*}

\begin{solution}
\begin{solutionsec}
    \item Let $\calV$ be an open cover of $X$. If $U \subseteq X$ is open $U \intersect V$ is open for all $V \in \calV$ (as Lee also remarks, this implication always holds). Conversely, if $U \intersect V$ is open in $V$ for all $V \in \calV$, then since each $V$ is open in $X$, $U \intersect V$ is also open in $X$. Furthermore, because $\calV$ is a cover of $X$ we have
    %
    \begin{equation*}
        U
            = U \intersect \bigunion_{V \in \calV} V
            = \bigunion_{V \in \calV} (U \intersect V),
    \end{equation*}
    %
    so $U$ is a union of open set, hence itself open.

    \item We first prove the following lemma:
    %
    \begin{displaytheorem}
        Let $\calF$ be a locally finite collection of closed sets in a topological space $X$. Then the union $\bbF = \bigunion_{F \in \calF} F$ is closed in $X$.
    \end{displaytheorem}
    %
    Let $x \in \bbF^c$. Then since $\calF$ is locally finite, $x$ has an open neighbourhood $U$ that intersects finitely many elements from $\calF$, say $F_1, \ldots, F_n$. Let $U' = U \setminus (F_1 \union \cdots \union F_n)$. Then $U'$ is an open neighbourhood of $x$ disjoint from $\bbF$, so $\bbF^c$ is open.
    
    We now solve the exercise. Let $\calF$ be a locally finite closed over of $X$, and let $C \subseteq X$ be such that $C \intersect F$ is closed in $F$ for all $F \in \calF$. Then
    %
    \begin{equation*}
        C
            = C \intersect \bigunion_{F \in \calF} F
            = \bigunion_{F \in \calF} (C \intersect F).
    \end{equation*}
    %
    The collection $\set{C \intersect F}{F \in \calF}$ is clearly also locally finite, so since each $C \intersect F$ is closed in $X$, the lemma shows that the above union is also closed in $X$.
\end{solutionsec}
\end{solution}

\end{document}