% Document setup
\documentclass[article, a4paper, 11pt, oneside]{memoir}
\usepackage[utf8]{inputenc}
\usepackage[T1]{fontenc}
\usepackage[UKenglish]{babel}

% Document info
\newcommand\doctitle{Lee: \emph{Introduction to Topological Manifolds}}
\newcommand\docauthor{Danny Nygård Hansen}

% Formatting and layout
\usepackage[autostyle]{csquotes}
\renewcommand{\mktextelp}{(\textellipsis\unkern)}
\usepackage[final]{microtype}
\usepackage{xcolor}
\frenchspacing
\usepackage{styles/articlepagestyle}
\usepackage{styles/articlesectionstyle}

% Fonts
\usepackage{amssymb}
\usepackage[largesmallcaps,partialup]{kpfonts}
\DeclareSymbolFontAlphabet{\mathrm}{operators} % https://tex.stackexchange.com/questions/40874/kpfonts-siunitx-and-math-alphabets
\linespread{1.06}
% \let\mathfrak\undefined
% \usepackage{eufrak}
\DeclareMathAlphabet\mathfrak{U}{euf}{m}{n}
\SetMathAlphabet\mathfrak{bold}{U}{euf}{b}{n}
% https://tex.stackexchange.com/questions/13815/kpfonts-with-eufrak
\usepackage{inconsolata}

% Hyperlinks
\usepackage{hyperref}
\definecolor{linkcolor}{HTML}{4f4fa3}
\hypersetup{%
	pdftitle=\doctitle,
	pdfauthor=\docauthor,
	colorlinks,
	linkcolor=linkcolor,
	citecolor=linkcolor,
	urlcolor=linkcolor,
	bookmarksnumbered=true
}

% Equation numbering
\numberwithin{equation}{chapter}

% Footnotes
\footmarkstyle{\textsuperscript{#1}\hspace{0.25em}}

% Mathematics
\usepackage{styles/basicmathcommands}
\usepackage{styles/framedtheorems}
\usepackage{styles/topologycommands}
\usepackage{tikz-cd}
\tikzcdset{arrow style=math font} % https://tex.stackexchange.com/questions/300352/equalities-look-broken-with-tikz-cd-and-math-font
\usetikzlibrary{babel}

% Lists
\usepackage{enumitem}
\setenumerate[0]{label=\normalfont(\alph*)}

% Bibliography
\usepackage[backend=biber, style=authoryear, maxcitenames=2, useprefix]{biblatex}
\addbibresource{references.bib}

% Title
\title{\doctitle}
\author{\docauthor}


%% Framed exercise environment

\mdfdefinestyle{swannexercise}{%
    skipabove=0.5em plus 0.4em minus 0.2em,
	skipbelow=0.5em plus 0.4em minus 0.2em,
	leftmargin=-5pt,
	rightmargin=-5pt,
	innerleftmargin=5pt,
	innerrightmargin=5pt,
	innertopmargin=5pt,
	innerbottommargin=4pt,
	linewidth=0pt,
	splittopskip=1.2em minus 0.2em,
	splitbottomskip=0.5em plus 0.2em minus 0.1em,
	backgroundcolor=backgroundcolor,
	frametitlebackgroundcolor=titlecolor,
	frametitlefont={\scshape},
    theoremseparator={\space\thechapter},
    theoremspace={.},
	frametitleaboveskip=3pt,
	frametitlebelowskip=2pt
}

\mdtheorem[style=swannexercise]{exerciseframed}{Exercise}

\let\oldexerciseframed\exerciseframed
\renewcommand{\exerciseframed}{%
  \crefalias{theorem}{exerciseframed}%
  \oldexerciseframed}


\mdtheorem[style=swannexercise]{problemframed}{Problem}

\let\oldproblemframed\problemframed
\renewcommand{\problemframed}{%
    \crefalias{theorem}{problemframed}%
    \oldproblemframed}

\makeatother



\theoremstyle{nonumberplain}
\theoremsymbol{\ensuremath{\square}}
\newtheorem{solution}{Solution}

\let\oldsolution\solution
\renewcommand{\solution}{%
  \crefalias{theorem}{solution}%
  \oldsolution}

\newcommand{\solutionlabelfont}[1]{{\normalfont\color{linkcolor}#1}}
\newlist{solutionsec}{enumerate}{1}
\setlist[solutionsec]{leftmargin=0pt, parsep=0pt, listparindent=\parindent, font=\solutionlabelfont, label=(\alph*), labelsep=0pt, labelwidth=20pt, itemindent=20pt, align=left, itemsep=10pt}

\newenvironment{displaytheorem}{%
	\begin{displayquote}\itshape%
}{%
	\end{displayquote}%
}


\newcommand{\calB}{\mathcal{B}}
\newcommand{\calV}{\mathcal{V}}
\newcommand{\calU}{\mathcal{U}}
\newcommand{\calF}{\mathcal{F}}
\newcommand{\bbF}{\mathbb{F}}

\begin{document}

\maketitle

\addtocounter{chapter}{4}
\chapter{Cell Complexes}

\begin{exerciseframed*}[3]
    Suppose $X$ is a topological space whose topology is coherent with a family $\calB$ of subspaces.
    %
    \begin{enumerate}
        \item If $Y$ is another topological space, then a map $f \colon X \to Y$ is continuous if and only if $f|_B$ is continuous for every $B \in \calB$.
        \item The map $q \colon \coprod_{B\in\calB} B \to X$ induced by the inclusion of each set $B \hookrightarrow X$ is a quotient map.
    \end{enumerate}
\end{exerciseframed*}

\begin{solution}
\begin{solutionsec}
    \item Let $V \subseteq Y$. Then $f\preim(V)$ is open if and only if $(f|_B)\preim(V) = f\preim(V) \intersect B$ is open in $B$ for all $B \in \calB$. But this precisely expresses that each $f|_B$ is continuous, so the claim follows.

    \item Notice that $q\preim(U) = U \intersect B$ for all $U \subseteq X$. Since the topology on $X$ is coherent with $\calB$, the set $U$ is open if and only if $U \intersect B$ is open for all $B \in \calB$. But this precisely expresses that $q$ is a quotient map.
\end{solutionsec}
\end{solution}


\section*{Problems}

\begin{problemframed*}[5]
    Suppose $X$ is a topological space and $\{X_\alpha\}$ is a family of subspaces whose union in $X$. Show that the topology of $X$ is coherent with the subspaces $\{X_\alpha\}$ if and only if it is the finest topology on $X$ for which all of the inclusion maps $i_\alpha \colon X_\alpha \hookrightarrow X$ are continuous.
\end{problemframed*}

\begin{solution}
    This follows immediately from the fact that $i_\alpha\preim(U) = U \intersect X_\alpha$ for all $U \subseteq X$.
\end{solution}


\begin{problemframed*}[6]
    Suppose $X$ is a topological space. Show that the topology of $X$ is coherent with each of the following collections of subspaces of $X$:
    %
    \begin{enumerate}
        \item Any open over of $X$.
        \item Any locally finite closed cover of $X$.
    \end{enumerate}
\end{problemframed*}

\begin{solution}
\begin{solutionsec}
    \item Let $\calV$ be an open cover of $X$. If $U \subseteq X$ is open $U \intersect V$ is open for all $V \in \calV$ (as Lee also remarks, this implication always holds). Conversely, if $U \intersect V$ is open in $V$ for all $V \in \calV$, then since each $V$ is open in $X$, $U \intersect V$ is also open in $X$. Furthermore, because $\calV$ is a cover of $X$ we have
    %
    \begin{equation*}
        U
            = U \intersect \bigunion_{V \in \calV} V
            = \bigunion_{V \in \calV} (U \intersect V),
    \end{equation*}
    %
    so $U$ is a union of open set, hence itself open.

    \item We first prove the following lemma:
    %
    \begin{displaytheorem}
        Let $\calF$ be a locally finite collection of closed sets in a topological space $X$. Then the union $\bbF = \bigunion_{F \in \calF} F$ is closed in $X$.
    \end{displaytheorem}
    %
    Let $x \in \bbF^c$. Then since $\calF$ is locally finite, $x$ has an open neighbourhood $U$ that intersects finitely many elements from $\calF$, say $F_1, \ldots, F_n$. Let $U' = U \setminus (F_1 \union \cdots \union F_n)$. Then $U'$ is an open neighbourhood of $x$ disjoint from $\bbF$, so $\bbF^c$ is open.
    
    We now solve the exercise. Let $\calF$ be a locally finite closed over of $X$, and let $C \subseteq X$ be such that $C \intersect F$ is closed in $F$ for all $F \in \calF$. Then
    %
    \begin{equation*}
        C
            = C \intersect \bigunion_{F \in \calF} F
            = \bigunion_{F \in \calF} (C \intersect F).
    \end{equation*}
    %
    The collection $\set{C \intersect F}{F \in \calF}$ is clearly also locally finite, so since each $C \intersect F$ is closed in $X$, the lemma shows that the above union is also closed in $X$.
\end{solutionsec}
\end{solution}

\end{document}